\documentclass{article}
\usepackage{graphicx} % Required for inserting images

\title{Appendix: Derivations}
\author{Sophie Li }
\date{May 2025}

\begin{document}

\maketitle

\section{Introduction}
Hi! These are some motivations and derivations I write up as I work on my project. No definite structure to this, but I think it's important as a researcher and computer scientist to understand the theory behind what you implement. 

\section{Cholesky Decomposition}
The Cholesky decomposition is a computational linear algebra technique. I am using it in the context of sampling a d-dimensional multi-variate gaussian, whose parameters are the means (mu) and the covariance matrix. 

Things get weird in the complex case. For simplicity, I'll assume the decomposition matrix $L$ is real-valued. Note, this is guaranteed when we $C$ is a symmetric positive-definite matrice (see section 3). 

\section{Positive Semi-definiteness}
As mentioned before, the Cholesky Decomposition requires a matrix that is positive semidefinite. Fortunately, the covariance matrix turns out to have this property so we don't need to check separately. 
In this section, I first give some definitions of positive semi-definite and prove that they are equivalent. 
\newline
I then show that the covariance matrix is positive semi-definite (abbrev: pos semi-def). 

\end{document}
